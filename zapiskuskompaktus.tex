\input{config.tex}
\begin{document} 

\begin{multicols}{5}
\setlength{\premulticols}{1pt}
\setlength{\postmulticols}{1pt}
\setlength{\multicolsep}{1pt}
\setlength{\columnsep}{2pt}

\section{Computational complexity}
\subsection{Tight bound $\Theta$}
\begin{align*}
	\Theta(g) = \{&f; \exists c_1,c_2,n_0>0, \\
	&\forall n>n_0: \\
	&0 \leq c_1g(n)\leq f(n) \leq c_2g(n)\}
\end{align*}
\subsection{Upper bound $\mathcal{O}$}
\begin{align*}
	\mathcal{O}(g) = \{&f; \exists c,n_0>0, \\
	&\forall n>n_0: \\
	&0 \leq f(n) \leq cg(n)\}
\end{align*}
\subsection{Lower bound $\Omega$}
\begin{align*}
	\Omega(g) = \{&f; \exists c,n_0>0, \\
	&\forall n>n_0: \\
	&0 \leq cg(n) \leq f(n)\}
\end{align*}
\subsection{Imprecise boundaries $o$ and $\omega$}
\[
	o(g) = \{f;\forall c>0,\exists n_0>0,\forall n > n_0: 0 \leq f(n) < cg(n)\}
\]
\[
	\omega(g) = \{f;\forall c>0,\exists n_0>0,\forall n > n_0: 0 \leq cg(n) < f(n)\}
\]
\subsection{Properties}
\begin{itemize}
	\item transitivity $f \in \Theta(g) \land g \in \Theta(h) \Rightarrow f \in \Theta(h)$ (for all bounds)
	\item reflexivity $f \in \Theta(f)$ (for $\Theta$, $\mathcal{O}$ and $\Omega$)
	\item symmetry $f \in \Theta(g) \Leftrightarrow g \in \Theta(f)$
	\item transpose symmetry $f \in \mathcal{O}(g) \Leftrightarrow g \in \Omega(f)$ \\
	$f \in o(g) \Leftrightarrow g \in \omega(f)$
\end{itemize}

\section{Divide and conquer}
\begin{itemize}
	\item \textbf{divide} the problem into several (equal) parts
	\item (recursively) \textbf{conquer (solve)} each of the sub problems
	\item \textbf{combine} sub problem solutions
\end{itemize}

\section{Simplified Masters}
\begin{align*}
	T(n) &= aT(\frac{n}{b})+\Theta(n^d) \\
	a &\geq 1 \\
	b &> 1 \\
	d &\geq 0
\end{align*}
\begin{itemize}
	\item $a > b^d \rightarrow T(n) = \Theta(n^{\log_ba})$
	\item $a = b^d \rightarrow T(n) = \Theta(n^d\log_bn)$
	\item $a < b^d \rightarrow T(n) = \Theta(n^d)$
\end{itemize}

\section{Masters}
\begin{align*}
	T(n) &= aT(\frac{n}{b})+f(n) \\
	a &\geq 1 \\
	b &> 1
\end{align*}
\begin{itemize}
	\item $f(n) = \mathcal{O}(n^{log_ba-\epsilon})\rightarrow T(n) = \Theta(n^{\log_ba}), \epsilon > 0$
	\item $f(n) = \Theta(n^{log_ba})\rightarrow T(n) = \Theta(n^{log_ba}\log(n))$
	\item $f(n) = \Omega(n^{log_ba+\epsilon})\rightarrow T(n) = \Theta(f(n)), \epsilon > 0$
		and $af(\frac{n}{b}) \leq cf(n)$ for some $c < 1$ and big enough $n$
	\item case2 ext: $f(n) = \Theta(n^{log_ba}log^k(n)) \rightarrow T(n) = \Theta(n^{log_ba}log^{k+1}(n))$
\end{itemize}


\section{Akra-Bazzi}
\[
	T(n) = \sum_{i=1}^ka_iT(b_in)+f(n), n > n_0
\]
\begin{itemize}
	\item $n_0 \geq \frac{1}{b_i}$, $n_0 \geq \frac{1}{1-b_i}$ for each $i$,
	\item $a_i > 0$ for each $i$,
	\item $0<b_i$ for each $i$,
	\item $k \geq 1$,
	\item $f(n)$ is non-negative function,
	\item $c_1f(n) \leq f(u) \leq c_2f(n)$, for each $u$ satisfying condition: $b_in\leq u\leq n$
\end{itemize}
\[
	T(n) = \Theta(n^p(1+\int_1^n\frac{f(u)}{u^{p+1}}du))
\]
we get $p$ from:
\[
	\sum_{i=1}^ka_ib_i^p=1
\]
\subsection{Extended Akra-Bazzi}
\[
	T(n) = \sum_{i=1}^ka_iT(b_in+h_i(n))+f(n), n > n_0
\]
all of the conditions from Akra-Bazzi still hold plus:
\[
	|h_i(n)| = \mathcal{O}(\frac{n}{\log^2n})
\]

\section{Annihilators}
Steps:
\begin{itemize}
	\item Write the recurrence in operator form.
	\item Extract the annihilator for the recurrence.
	\item Factor the annihilator (if necessary).
	\item Extract the generic solution form the annihilator.
	\item Solve for coefficients using base cases (if known).
\end{itemize}
\setlength{\tabcolsep}{0.5em}{\renewcommand{\arraystretch}{1.2}
\begin{tabular}{ | c | c | }
    \hline
    \textbf{Operator} & \textbf{Definition}\\\hline
    addition & $(f+g)(n) := f(n)+g(n)$ \\
    subtraction & $(f-g)(n) := f(n)-g(n)$ \\
    multiplication & $(a\cdot f)(n) := a\cdot (f(n))$ \\
    shift & $E f(n) := f(n+1)$ \\
    $k$-fold shift & $E^k f(n) := f(n+k)$ \\ \hline
    composition & $(X+Y)f:= Xf+Yf$ \\
    & $(X-Y)f:= Xf-Yf$ \\
    & $XYf:= X(Yf)=Y(Xf)$ \\
    distribution & $X(f+g)= Xf+Xg$ \\\hline
\end{tabular}
}
\setlength{\tabcolsep}{0.5em}{\renewcommand{\arraystretch}{1.2}
\begin{tabular}{ | c | l | }
    \hline
    \textbf{Operator} & \textbf{Functions annihilated}\\\hline
    $E -1$ & $\alpha$ \\
    $E -a$ & $\alpha a^n$ \\
    $(E-a)(E-b)$ & $\alpha a^n+\beta b^n$\hfill ($a \neq b$) \\
    $(E-a_0)(E-a_1)\cdots(E-a_k)$ & $\sum_{i=0}^k\alpha_ia_i^n$\hfill ($a_i$ distinct) \\
    $(E - 1)^2$ & $\alpha n + \beta$ \\
    $(E - a)^2$ & $(\alpha n + \beta)a^n$ \\
    $(E - a)^2(E-b)$ & $(\alpha n + \beta)a^n+\gamma b^n$\hfill ($a\neq b$)\\
    $(E - a)^d$ & $(\sum_{i=0}^{d-1}\alpha_in^i)a^n$ \\\hline\hline
    \multicolumn{2}{|c|}{If $X$ annihilates $f$, then $X$ also annihilates $E f$.} \\
\multicolumn{2}{|c|}{If $X$ annihilates both $f$ and $g$,} \\
\multicolumn{2}{|c|}{then $X$ also annihilates $f\pm g$.} \\
\multicolumn{2}{|c|}{If $X$ annihilates $f$, then $X$ also annihilates $\alpha f$,} \\
\multicolumn{2}{|c|}{for any constant $\alpha$.} \\\hline\hline
    \multicolumn{2}{|c|}{If $X$ annihilates $f$ and $Y$ annihilates $g$,}\\
    \multicolumn{2}{|c|}{then $XY$ annihilates $f\pm g$.} \\\hline
\end{tabular}
}
\section{Randomization}
To avoid bad input sequences, the input can be intentionally randomized.
\section{Pseudo random generator}
\subsection{Linear congruential generators}
\[
	x_i = (a x_{i-1} + c) \mod m
\]
\begin{itemize}
	\item RANDU: $x_i = 65539 x_{i-1}\pmod {2^{31}}$
	\item MINSTD $x_i = 16807 x_{i-1}\pmod {2^{31}-1}$
	\item Combinations of linear congruential generators. Addition, subtraction, bit mixing.
		Better randomness, small period.
	\item higher order recursions
\end{itemize}
\subsection{Blum-Blum-Shrub}
\begin{itemize}
	\item $p, q\in \mathbb{P}$, large (at least 40 decimal places)
	\item $m = pq$
	\item $X_i=X_{i-1}^2 \pmod m$
	\item $b_i = \text{parity}(X_i)$
\end{itemize}


\section{Amortized analysis}
\subsection{Aggregated analysis}
Determine upper bound $T(n)$ for the total cost of a sequence of $n$ operations.
Amortized cost per operation is $\frac{T(n)}{n}$.
\subsection{Accounting method}
Some operations are overcharged to pay for other operations.
\subsection{Potential method}
Potential function is tied to a data structure.
\section{NP-complete problems}
\begin{itemize}
	\item CSAT -- logical circuit satisfiability
	\item FSAT -- logical formula satisfiability
	\item 3CNF-SAT -- formula in 3-conjuctive normal form satisfiability
	\item CLIQUE -- existence of cliques in a graph
	\item VERTEX COVER -- a minimal set of vertices that cover all the edges of a graph
	\item HAM -- Hamiltonian cycle of a graph
	\item TSP -- travelling salesman problem
	\item SUBSET-SUM -- the subset of numbers equal to a given number
	\item BIN-TREE -- optimal binary decision tree
	\item SUBGRAPH-ISOMORPHISM
\end{itemize}
\section{Linear programming}
\subsection{Standard LP}
\begin{itemize}
	\item given $n$ real numbers $c_1,c_2,\ldots,c_n$
	\item $m$ real numbers $b_1,b_2,\ldots,b_m$
	\item $m\times n$ real numbers $a_{ij}$ for $i=1,\ldots,m$ and $j=1,\ldots,n$
	\item we wish to find $n$ real numbers $x_1,\ldots,x_n$ that
\end{itemize}
maximize $\sum_{j=1}^nc_jx_j$ subject to
\[
	\sum_{j=1}^na_{ij}x_j\leq b_i, \forall i=1,\ldots,m
\]
\[x_j\geq 0\]
\subsection{Approximation}
LP relaxation, 0-1 integer programming
\section{Local search}
\begin{itemize}
	\item State space: $S=\{S;S_Z\longrightarrow S\}$
	\item starting state: $S_0$
	\item quality of state: $q(S)$
	\item global optimum: $S_{\text{best}} = \arg \min_{s\in S} q(s)$
	\item local optima: $S_{\text{local}} = \{S;\forall S\rightarrow S\prime : q(S) \leq q(S\prime)\}$
\end{itemize}
\subsection{Problems}
\begin{itemize}
	\item local extremes
	\item plato
	\item ridge
\end{itemize}
\subsection{Metroplis algorithm}
\begin{itemize}
	\item If better neighbour exists, move to it.
	\item Otherwise choose a random neighbour, but accept better neighbours with larger probability.
	\item Decrease the probability of acceptance.
	\item In time, stohastic search turns into deterministic LS.
\end{itemize}
\subsection{Simulated annealing}
\begin{itemize}
	\item Start with a random state $S$.
	\item Select random neighbour $S\prime$
	\item If $q(S\prime) < q(S)$, move to $S\prime$.
	\item Otherwise, move with probability $e^\frac{-(q(S\prime)-q(S))}{T}$
\end{itemize}
Decrease temperature while it's not close to zero.
Usually a geometrical rule is used: $T\prime = \lambda T$, $0<\lambda<1$ (typically $\lambda = 0.95$)

\end{multicols}
\end{document}
