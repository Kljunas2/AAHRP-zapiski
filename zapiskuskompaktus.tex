\documentclass[a3paper,8pt]{extarticle}
% \usepackage[utf8]{inputenc}
\usepackage[mathletters]{ucs}
\usepackage[utf8x]{inputenc}

\usepackage{fancyhdr}

\usepackage[pdftex]{graphicx} % Required for including pictures
\usepackage[pdftex,linkcolor=black,pdfborder={0 0 0}]{hyperref} % Format links for pdf
\usepackage{calc} % To reset the counter in the document after title page
\usepackage{enumitem} % Includes lists

\usepackage{textcomp}
\usepackage{eurosym}

\usepackage{ dsfont } % font za množice
% tabele
\usepackage{array}
\usepackage{wrapfig}

\usepackage{tikz,forest}
\usetikzlibrary{arrows.meta}

\frenchspacing % No double spacing between sentences
\setlength{\parindent}{0pt}
\setlength{\parskip}{0.1em}

\usepackage{mathtools}
\usepackage{blkarray, bigstrut} %


\usepackage{amssymb,amsmath,amsthm,amsfonts}
\usepackage{multicol,multirow}
\usepackage{calc}
\usepackage{ifthen}
\usepackage{tabularx}
\usepackage[landscape]{geometry}
\usepackage{listings}
\usepackage{inconsolata}
%\usepackage[colorlinks=true,citecolor=blue,linkcolor=blue]{hyperref}
%\usepackage{accents}

\newcommand{\vect}[1]{\accentset{\rightharpoonup}{#1}}

\ifthenelse{\lengthtest { \paperwidth = 11in}}
    { \geometry{top=.5in,left=.5in,right=.5in,bottom=.5in} }
	{\ifthenelse{ \lengthtest{ \paperwidth = 297mm}}
		{\geometry{top=1cm,left=1cm,right=1cm,bottom=1cm} }
		{\geometry{top=1cm,left=1cm,right=1cm,bottom=1cm} }
	}
\pagestyle{empty}
\makeatletter
\renewcommand{\section}{\@startsection{section}{1}{0mm}%
                                {-1ex plus -.5ex minus -.2ex}%
                                {0.5ex plus .2ex}%x
                                {\normalfont\large\bfseries}}
\renewcommand{\subsection}{\@startsection{subsection}{2}{0mm}%
                                {-1explus -.5ex minus -.2ex}%
                                {0.5ex plus .2ex}%
                                {\normalfont\normalsize\bfseries}}
\renewcommand{\subsubsection}{\@startsection{subsubsection}{3}{0mm}%
                                {-1ex plus -.5ex minus -.2ex}%
                                {1ex plus .2ex}%
                                {\normalfont\small\bfseries}}
\makeatother
\setcounter{secnumdepth}{0}
%\setlength{\parindent}{0pt}
%\setlength{\parskip}{0pt plus 0.5ex}

% listings okolje za psevdo kodo
\lstnewenvironment{koda}[1][] %defines the algorithm listing environment
{   
    \lstset{ %this is the stype
        mathescape=true,
        basicstyle=\scriptsize, 
		columns=flexible,
        keywordstyle=\bfseries\em,
        keywords={,vhod, izhod, zacetek, konec, koncamo, ponavljaj, dokler, ce, vrni, za, vsak, vse, v, sicer,} %add the keywords you want, or load a language as Rubens explains in his comment above.
        xleftmargin=.1\textwidth,
		tabsize=4,
		%frame=leftline,xleftmargin=5pt,xrightmargin=5pt,framesep=5pt,
		%inputencoding = utf8,
		extendedchars = true,
		literate={ž}{{\ˇz}}1 {š}{{\ˇs}}1 {č}{{\ˇc}}1 {Ž}{{\ˇZ}}1 {Š}{{\ˇS}}1 {Č}{{\ˇC}}1,
        #1 % this is to add specific settings to an usage of this environment (for instnce, the caption and referable label)
    }
}
{}
% -----------------------------------------------------------------------

\begin{document} 

\begin{multicols}{5}
\setlength{\premulticols}{1pt}
\setlength{\postmulticols}{1pt}
\setlength{\multicolsep}{1pt}
\setlength{\columnsep}{2pt}

\section{Computational complexity}
\subsection{Tight bound $\Theta$}
\begin{align*}
	\Theta(g) = \{&f; \exists c_1,c_2,n_0>0, \\
	&\forall n>n_0: \\
	&0 \leq c_1g(n)\leq f(n) \leq c_2g(n)\}
\end{align*}
\subsection{Upper bound $\mathcal{O}$}
\begin{align*}
	\mathcal{O}(g) = \{&f; \exists c,n_0>0, \\
	&\forall n>n_0: \\
	&0 \leq f(n) \leq cg(n)\}
\end{align*}
\subsection{Lower bound $\Omega$}
\begin{align*}
	\Omega(g) = \{&f; \exists c,n_0>0, \\
	&\forall n>n_0: \\
	&0 \leq cg(n) \leq f(n)\}
\end{align*}
\subsection{Imprecise boundaries $o$ and $\omega$}
\[
	o(g) = \{f;\forall c>0,\exists n_0>0,\forall n > n_0: 0 \leq f(n) < cg(n)\}
\]
\[
	\omega(g) = \{f;\forall c>0,\exists n_0>0,\forall n > n_0: 0 \leq cg(n) < f(n)\}
\]
\subsection{Properties}
\begin{itemize}
	\item transitivity $f \in \Theta(g) \land g \in \Theta(h) \Rightarrow f \in \Theta(h)$ (for all bounds)
	\item reflexivity $f \in \Theta(f)$ (for $\Theta$, $\mathcal{O}$ and $\Omega$)
	\item symmetry $f \in \Theta(g) \Leftrightarrow g \in \Theta(f)$
	\item transpose symmetry $f \in \mathcal{O}(g) \Leftrightarrow g \in \Omega(f)$ \\
	$f \in o(g) \Leftrightarrow g \in \omega(f)$
\end{itemize}

\section{Divide and conquer}
\begin{itemize}
	\item \textbf{divide} the problem into several (equal) parts
	\item (recursively) \textbf{conquer (solve)} each of the sub problems
	\item \textbf{combine} sub problem solutions
\end{itemize}

\section{Simplified Masters}
\begin{align*}
	T(n) &= aT(\frac{n}{b})+\Theta(n^d) \\
	a &\geq 1 \\
	b &> 1 \\
	d &\geq 0
\end{align*}
\begin{itemize}
	\item $a > b^d \rightarrow T(n) = \Theta(n^{\log_ba})$
	\item $a = b^d \rightarrow T(n) = \Theta(n^d\log_bn)$
	\item $a < b^d \rightarrow T(n) = \Theta(n^d)$
\end{itemize}

\section{Masters}
\begin{align*}
	T(n) &= aT(\frac{n}{b})+f(n) \\
	a &\geq 1 \\
	b &> 1
\end{align*}
\begin{itemize}
	\item $f(n) = \mathcal{O}(n^{log_ba-\epsilon})\rightarrow T(n) = \Theta(n^{\log_ba}), \epsilon > 0$
	\item $f(n) = \Theta(n^{log_ba})\rightarrow T(n) = \Theta(n^{log_ba}\log(n))$
	\item $f(n) = \Omega(n^{log_ba+\epsilon})\rightarrow T(n) = \Theta(f(n)), \epsilon > 0$
		and $af(\frac{n}{b}) \leq cf(n)$ for some $c < 1$ and big enough $n$
	\item case2 ext: $f(n) = \Theta(n^{log_ba}log^k(n)) \rightarrow T(n) = \Theta(n^{log_ba}log^{k+1}(n))$
\end{itemize}


\section{Akra-Bazzi}
\[
	T(n) = \sum_{i=1}^ka_iT(b_in)+f(n), n > n_0
\]
\begin{itemize}
	\item $n_0 \geq \frac{1}{b_i}$, $n_0 \geq \frac{1}{1-b_i}$ for each $i$,
	\item $a_i > 0$ for each $i$,
	\item $0<b_i$ for each $i$,
	\item $k \geq 1$,
	\item $f(n)$ is non-negative function,
	\item $c_1f(n) \leq f(u) \leq c_2f(n)$, for each $u$ satisfying condition: $b_in\leq u\leq n$
\end{itemize}
\[
	T(n) = \Theta(n^p(1+\int_1^n\frac{f(u)}{u^{p+1}}du))
\]
we get $p$ from:
\[
	\sum_{i=1}^ka_ib_i^p=1
\]
\subsection{Extended Akra-Bazzi}
\[
	T(n) = \sum_{i=1}^ka_iT(b_in+h_i(n))+f(n), n > n_0
\]
all of the conditions from Akra-Bazzi still hold plus:
\[
	|h_i(n)| = \mathcal{O}(\frac{n}{\log^2n})
\]

\section{Annihilators}
Steps:
\begin{itemize}
	\item Write the recurrence in operator form.
	\item Extract the annihilator for the recurrence.
	\item Factor the annihilator (if necessary).
	\item Extract the generic solution form the annihilator.
	\item Solve for coefficients using base cases (if known).
\end{itemize}
\setlength{\tabcolsep}{0.5em}{\renewcommand{\arraystretch}{1.2}
\begin{tabular}{ | c | c | }
    \hline
    \textbf{Operator} & \textbf{Definition}\\\hline
    addition & $(f+g)(n) := f(n)+g(n)$ \\
    subtraction & $(f-g)(n) := f(n)-g(n)$ \\
    multiplication & $(a\cdot f)(n) := a\cdot (f(n))$ \\
    shift & $E f(n) := f(n+1)$ \\
    $k$-fold shift & $E^k f(n) := f(n+k)$ \\ \hline
    composition & $(X+Y)f:= Xf+Yf$ \\
    & $(X-Y)f:= Xf-Yf$ \\
    & $XYf:= X(Yf)=Y(Xf)$ \\
    distribution & $X(f+g)= Xf+Xg$ \\\hline
\end{tabular}
}
\setlength{\tabcolsep}{0.5em}{\renewcommand{\arraystretch}{1.2}
\begin{tabular}{ | c | l | }
    \hline
    \textbf{Operator} & \textbf{Functions annihilated}\\\hline
    $E -1$ & $\alpha$ \\
    $E -a$ & $\alpha a^n$ \\
    $(E-a)(E-b)$ & $\alpha a^n+\beta b^n$\hfill ($a \neq b$) \\
    $(E-a_0)(E-a_1)\cdots(E-a_k)$ & $\sum_{i=0}^k\alpha_ia_i^n$\hfill ($a_i$ distinct) \\
    $(E - 1)^2$ & $\alpha n + \beta$ \\
    $(E - a)^2$ & $(\alpha n + \beta)a^n$ \\
    $(E - a)^2(E-b)$ & $(\alpha n + \beta)a^n+\gamma b^n$\hfill ($a\neq b$)\\
    $(E - a)^d$ & $(\sum_{i=0}^{d-1}\alpha_in^i)a^n$ \\\hline\hline
    \multicolumn{2}{|c|}{If $X$ annihilates $f$, then $X$ also annihilates $E f$.} \\
\multicolumn{2}{|c|}{If $X$ annihilates both $f$ and $g$,} \\
\multicolumn{2}{|c|}{then $X$ also annihilates $f\pm g$.} \\
\multicolumn{2}{|c|}{If $X$ annihilates $f$, then $X$ also annihilates $\alpha f$,} \\
\multicolumn{2}{|c|}{for any constant $\alpha$.} \\\hline\hline
    \multicolumn{2}{|c|}{If $X$ annihilates $f$ and $Y$ annihilates $g$,}\\
    \multicolumn{2}{|c|}{then $XY$ annihilates $f\pm g$.} \\\hline
\end{tabular}
}
\section{Randomization}
To avoid bad input sequences, the input can be intentionally randomized.
\section{Pseudo random generator}
\subsection{Linear congruential generators}
\[
	x_i = (a x_{i-1} + c) \mod m
\]
\begin{itemize}
	\item RANDU: $x_i = 65539 x_{i-1}\pmod {2^{31}}$
	\item MINSTD $x_i = 16807 x_{i-1}\pmod {2^{31}-1}$
	\item Combinations of linear congruential generators. Addition, subtraction, bit mixing.
		Better randomness, small period.
	\item higher order recursions
\end{itemize}
\subsection{Blum-Blum-Shrub}
\begin{itemize}
	\item $p, q\in \mathbb{P}$, large (at least 40 decimal places)
	\item $m = pq$
	\item $X_i=X_{i-1}^2 \pmod m$
	\item $b_i = \text{parity}(X_i)$
\end{itemize}


\section{Amortized analysis}
\subsection{Aggregated analysis}
Determine upper bound $T(n)$ for the total cost of a sequence of $n$ operations.
Amortized cost per operation is $\frac{T(n)}{n}$.
\subsection{Accounting method}
Some operations are overcharged to pay for other operations.
\subsection{Potential method}
Potential function is tied to a data structure.
\section{NP-complete problems}
\begin{itemize}
	\item CSAT -- logical circuit satisfiability
	\item FSAT -- logical formula satisfiability
	\item 3CNF-SAT -- formula in 3-conjuctive normal form satisfiability
	\item CLIQUE -- existence of cliques in a graph
	\item VERTEX COVER -- a minimal set of vertices that cover all the edges of a graph
	\item HAM -- Hamiltonian cycle of a graph
	\item TSP -- travelling salesman problem
	\item SUBSET-SUM -- the subset of numbers equal to a given number
	\item BIN-TREE -- optimal binary decision tree
	\item SUBGRAPH-ISOMORPHISM
\end{itemize}
\section{Linear programming}
\subsection{Standard LP}
\begin{itemize}
	\item given $n$ real numbers $c_1,c_2,\ldots,c_n$
	\item $m$ real numbers $b_1,b_2,\ldots,b_m$
	\item $m\times n$ real numbers $a_{ij}$ for $i=1,\ldots,m$ and $j=1,\ldots,n$
	\item we wish to find $n$ real numbers $x_1,\ldots,x_n$ that
\end{itemize}
maximize $\sum_{j=1}^nc_jx_j$ subject to
\[
	\sum_{j=1}^na_{ij}x_j\leq b_i, \forall i=1,\ldots,m
\]
\[x_j\geq 0\]
\subsection{Approximation}
LP relaxation, 0-1 integer programming
\section{Local search}
\begin{itemize}
	\item State space: $S=\{S;S_Z\longrightarrow S\}$
	\item starting state: $S_0$
	\item quality of state: $q(S)$
	\item global optimum: $S_{\text{best}} = \arg \min_{s\in S} q(s)$
	\item local optima: $S_{\text{local}} = \{S;\forall S\rightarrow S\prime : q(S) \leq q(S\prime)\}$
\end{itemize}
\subsection{Problems}
\begin{itemize}
	\item local extremes
	\item plato
	\item ridge
\end{itemize}
\subsection{Metroplis algorithm}
\begin{itemize}
	\item If better neighbour exists, move to it.
	\item Otherwise choose a random neighbour, but accept better neighbours with larger probability.
	\item Decrease the probability of acceptance.
	\item In time, stohastic search turns into deterministic LS.
\end{itemize}
\subsection{Simulated annealing}
\begin{itemize}
	\item Start with a random state $S$.
	\item Select random neighbour $S\prime$
	\item If $q(S\prime) < q(S)$, move to $S\prime$.
	\item Otherwise, move with probability $e^\frac{-(q(S\prime)-q(S))}{T}$
\end{itemize}
Decrease temperature while it's not close to zero.
Usually a geometrical rule is used: $T\prime = \lambda T$, $0<\lambda<1$ (typically $\lambda = 0.95$)
\section{Metaheuristics}
\subsection{Tabu search}
Idea: to prevent returning back to the same local extreme, supress (parts of) solutions.
\subsection{Guided local search}
Metaheuristics which guide local search and helps it avoid local extremes.
\begin{itemize}
	\item define properties (attributes) of solutions
	\item penalize attributes, which occur too often in local extrema
	\item auxiliary objective function
		\[h(s) = g(s) + \lambda \cdot \sum_{i\text{ is a feature}}(p_i\cdot I_i(s))\]
\end{itemize}
Utility of punishment for property $i$ in local extreme $s*$
\[
	\text{util}_i(s*) = I_i(s*) \cdot \frac{c_i}{1+p}
\]
$c_i$ is cost, $p_i$ is current punishment for property $i$ \\
In local extreme we punish the property with the largest utility (we increment $p_i$ by 1).
\subsection{Variable neighbourhood search}
Idea: define several neighbourhood structures and change neighbourhood when reaching local extreme
in one of them. Order neighbourhoods by the efficiency of computation.

\section{Swarm intelligence}
\begin{itemize}
	\item fixed population
	\item autonomous individual
	\item communication between agents
	\item aggregation of similar animals, generally cruising in the same direction
	\item simple rules for each individual
	\item decentralized
	\item emergent behaviour
\end{itemize}
\subsection{Ant colony optimization}
\begin{itemize}
	\item ants find the shortest path to food source from the nest
	\item they deposit pheromone along traveled path, which is used by other ants to follow the trail
	\item this kind of indirect communication via the local environment is called stigmergy
	\item adaptability, robustness and redundancy
\end{itemize}
Possible daemon actions to apply centralized actions.
\subsection{Particle swarm optimization}
\begin{itemize}
	\item Individuals strive to improve themselves and often achieve this by observing and imitating their neighbours.
	\item Each individual has the ability to remember.
	\item Each particle is represented with two vectors, location and velocity.
\end{itemize}
\subsubsection{Information exchange in the swarm}
\begin{itemize}
	\item historically best location $x^*$
	\item best location of informants $x^+$
	\item globally best location $x^!$
\end{itemize}
\subsubsection{Moving}
\begin{itemize}
	\item Compute the fitness of each particle and update $x^*$, $x^+$ and $x^!$.
	\item Update the representation of particle.
		Velocity vector takes into account updated directions $x^*$, $x^+$ and $x^!$.
		Each direction is updated with some random noise.
	\item Move the particle in the direction of the velocity vector.
\end{itemize}


\end{multicols}
\end{document}
